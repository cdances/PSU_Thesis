%SourceDoc ../YourName-Dissertation.tex
\vspace*{-80mm}
\chapter{Introduction} \label{chapter1:introduction}

For the past several decades, the primary focus in nuclear engineering within
the United States has been on light water reactors (LWR). Commercially,
all nuclear reactors are either boiling water reactors (BWR) or pressurized
water reactors (PWR). Correct computation of the thermal hydraulics within the
reactor core leads to efficient design and accuracy in the safety analysis. 
CASL is a key player in the effort, and utilizes the popular subchannel code,
CTF, for modeling the hydrodynamics within the reactor core \cite{Schmidt2014}.
This FORTRAN based code developed from COBRA-TF solves 8 conservation equations
for liquid, entrained droplet, and vapor phases phases, plus one conservation
equation for non-condensable gases \cite{CTF_theory}. 

The set of procedures for ensuring that simulation codes such as CTF are
accurate and reliable is called software validation and verification
\cite{Oberkampf2008}. CTF has undergone software uncertainty quantification and
benchmark validation \cite{Avramova2015}. The current version of CTF has
standard verification practices that focus on software quality engineering
similar to those in other versions of COBRA-TF \cite{Aumiller2013}, but
currently lacks an in depth verification document that focuses on numerical
algorithm verification. This work focuses on this second type of verification
for the original version of CTF as well as a residual formulation. 

The 1-D residual formulation of the code has been created for single phase
liquid coupled to radial conduction. While other residual formulations have been
formed for other versions of COBRA-TF \cite{Lloyd2014}, none have been
integrated into the CASL version of CTF. This work details the verification of
both the residual formulation and the original version of CTF for the single
phase liquid and radial conduction equations. The verification problems will
study of the order of accuracy of the errors, which is considered one of the
more rigorous verification criteria \cite{Roy2005}. This work will be
considered a starting point for future work to perform verification on the
single phase equations in both axial and transverse dimensions
\cite{Merroun2009}, and two phase flow \cite{Mahadevan2009}.

%\section{CTF}