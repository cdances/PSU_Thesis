%SourceDoc ../YourName-Dissertation.tex
\vspace*{-80mm}
\chapter{Introduction} \label{chapter1:introduction}

For the past several decades, the primary focus in nuclear engineering
within the United States has been focused on light water reactors (LWR).
Commercially, all nuclear reactors are either boiling water reactors (BWR)
or pressurized water reactors (PWR). Correct computation of the
thermal hydraulics within the reactor core leads to efficient design and
accuracy in the safety analysis. A popular subchannel code for modelling the
hydrodynamics with in the reactor core is COBRA-TF.
This FORTRAN based code solves 8 conservation equations for liquid,
entrained droplet, and vapor phases in 3-D dimmensions \cite{CTF_Theory}.
The conservation equations analytically reduce into a pressure matrix in a
semi-implicit  method with rod temperatures solved for explicitly. Because
the physics are integrated into the numerical solution, the equations must
be linear and the solution method semi-implicit. With a residual
formulation, greater flexibility and control over the numerical solution
is possible. COBRA-TF was originally written in FORTRAN 77, but over
the years has been partially updated to newer versions of Fortran.









