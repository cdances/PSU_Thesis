%SourceDoc ../YourName-Dissertation.tex
\vspace*{-80mm}
\chapter{Introduction} \label{chapter1:introduction}

For the past several decades, the primary focus in nuclear engineering within
the United States has been on light water reactors (LWR). Commercially,
all nuclear reactors are either boiling water reactors (BWR) or pressurized
water reactors (PWR). Correct computation of the thermal hydraulics within the
reactor core leads to efficient design and accuracy in the safety analysis. A
popular subchannel code for modeling the hydrodynamics within the reactor core
is CTF, which is a subchannel thermal-hydraulics code developed from
COBRA-TF \cite{Salko2014}. This FORTRAN based code solves 8 conservation
equations for liquid, entrained droplet, and vapor phases phases, plus one
conservation equation for non-condensable gases. A 1-D residual formulation of
the code has been created. While other residual formulations have been
formed for other versions of COBRA-TF \cite{Lloyd2014}, none have been
integrated into the CASL version of CTF. The current version of CTF has standard
verification practices that focus on software quality engineering similar to
those in other versions of COBRA-TF \cite{Aumiller2013}, but lacks an in
depth verification document that focuses on numerical algorithm verification.
This paper focues on this second type of verification and outlines the initial
verification of the original version of the code as well as the residual version
of the code. The verification problem is a single phase 1-D channel with
transient inlet density and mass flow rate. The problem will undergo a
Richardson's extrapolation in the temporal and spatial domains to verify the
convergence and order of accuracy of the error. The study of the order of
accuracy is considered one of the more rigorous verification criteria \cite{Roy2005}.
This work will be expanded to perform verificaiton on the single
phase equations in both axial and transverse dimensions \cite{Merroun2009}, and
coupled fluid heat conduction \cite{Mahadevan2009}.









