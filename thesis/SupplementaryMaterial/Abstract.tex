% Place abstract below.
Nuclear engineering codes  are being used to simulate more challenging problems
and at higher fidelities  than they were initially developed for. In order to
expand the capabilities of  these codes, state of the art numerical methods and
computer science need to be  implemented. One of the key players in this effort
is the Consortium for Advanced  Simulation of Light Water Reactors (CASL) and
through development of the  Virtual Environment for Reactor Applications (VERA).
The sub-channel thermal  hydraulic code used in VERA, COBRA-TF (Coolant-Boiling
in Rod Arrays - Three  Fluids), is partially developed at the Pennsylvania State
University by the Reactor  Dynamics and Fuel Management Research Group (RDFMG).

Currently, COBRA-TF solves 8  conservation equations for liquid, entrained
droplet, and vapor phases of  water boiling within the rod structure of a LWR
reactor core. The conservation  equations analytically reduce into a pressure
matrix and are solved using a  semi-implicit method. The solid conduction
equations are then implicitly  solved to determine the temperature within the
fuel. Since the liquid solution  is solved independent of the solid solution,
the solid and liquid equations are explicitly coupled.

In an effort to help meet the objectives of CASL,  a version of COBRTA-TF has
been developed that solves the residual formulation  of the 1D single-phase
conservation equations. The formulation of the base  equations as residuals
allows the code to be run semi-implicitly or fully  implicitly while clearly
defining the original conservation equations. This  paper outlines work to
integrate 1D solid conduction equations into the  residual formulation. This
expands the solid liquid coupling to be either  explicit or implicit. Different
physical models, such as the homogeneous liquid  solid energy model, can be
readily implemented by adding the residual  functions and variables. A simple
test problem consisting of a single liquid  channel and fuel pin was designed to
compare the original version of COBRA-TF to  the different numerical and
physical models available through the new  residual formulation. The methods are
compared both for steady state and transient  conditions to quantify the
accuracy and stability of each method. The input  parameters are varied over a
variety of conditions to demonstrate when different  methods are most
appropriate. The ability to choose appropriate numerical  methods and physical
models will allow for greater fidelity, decrease computational expenses.
