%SourceDoc ../YourName-Dissertation.tex
\vspace*{-80mm}
\chapter{Conclusions} \label{chapter7:uniform_heating}
 
%Round off error, iterative convergence error.
%Discretization error?? (Check with)

The residual formulation of CTF allows for a numerical computation of the
multivariable Jacobian matrix and produces comparable results to the original
code which uses an analytical derivation of a pressure matrix. The 1-D
single phase liquid verification problems were able to
isolate of the order of accuracies for the conservation equations and show that
they match the values obtained from the modified equation analysis. The
discretization error for both versions of the code converged to zero with
decreasing time step and axial mesh size. The order of accuracy for the temporal
and spatial refinements matched very closely with the modified equation analysis
for both codes. For all of these data points, the residual formulation of the
code showed discretization errors that were very close with the original version
of the code. The numerical error obtained from the output matched closely to
the analytical error predicted by the modified equation analysis using the
derivatives of the known solutions while within the asymptotic range. Future
work will involve adding in form losses, transverse flow, and multiple phases.

Combining the liquid and solid equations into a single Jacobian matrix allowed
for easy explicit or implicit coupling. This solution method was tested against
the analytical solution for a single rod with uniform heat generation and shown
to give good results for both the residual formulation and the original version
of CTF. The discrepancy between the results from CTF and the analytical solution
were shown to be dependent on the radial mesh size. Future verification work
will involve performing a Richardson extrapolation to obtain an order of
accuracy. A modified equation analysis of the explicit and implicit coupling
methods would also prove very useful. The effect of temperature dependent
material properties and dynamic gap conductance should also be considered. A
homogeneous energy equation can now be easily implemented by adding the liquid
and solid conservation equations. Future work will be analyzing the homogeneous
energy approximation over a state space to see when the approximation is valid.
Future work will also involve extending the conduction equations to azimuthal
and axial directions.


