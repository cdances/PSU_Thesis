%SourceDoc ../YourName-Dissertation.tex
\vspace*{-80mm}
\chapter{Conclusions} \label{chapter7:uniform_heating}
 
%Round off error, iterative convergence error.
%Discretization error?? (Check with)

The residual formulation of CTF allows for a numerical computation of the
multivariable Jacobian matrix compared to the original analytical derivation of
a pressure matrix. The 1-D isokinetic single phase liquid verification problem
is a good verification problem due its isolation of the order of accuracies
through modified equation analysis. The discretization error for both versions
of the code converged to zero with decreasing time step and axial mesh size.
The order of accuracy for the temporal and spatial refinements matched very
closely with the modified equation analysis for both codes. For all of these
data points, the residual formulation of the code showed discretization errors
that were very close with the original version of the code. Future work should
compare the numerical error obtained in the code to the analytical error
predicted by the modified equation analysis using the derivatives of the known
solutions. While within the asymptotic range, the first order accurate
analytical error should almost exactly match the error from the code.

The residual formulation of the one-dimensional single-phase liquid and solid
residual formulations were listed. Combining the liquid and solid equations into
a single Jacobian matrix allowed for easy explicit or implicit coupling. This
solution method was tested against the analytical solution for a single rod with
uniform heat generation and shown to give good results. Future work will involve
performing a more in depth verification analysis of the steady state and
transient solutions. The effect of temperature dependent material properties
and dynamic gap conductance will also be considered. A homogenous energy
equation can now be easily implemented by adding the liquid and solid
conservation equations. Future work will be analyzing the homogeneous energy
approximation over a state space to see when the approximation is valid. Future
work will also involve extending the conduction equations to azimuthal and axial
directions.







