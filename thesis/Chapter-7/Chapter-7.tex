%SourceDoc ../YourName-Dissertation.tex
\vspace*{-80mm}
\chapter{Shock Tube} \label{chapter7:Shock_Tube}
 
	\section{Problem Setup} \label{Verification:Shock_Tube}
        
    A shock tube is a very common and standard method of verification for
    momentum and pressure. However, an exact analytical solution is more readily
    obtained for an ideal gas such as air. A shock tube is created by setting
    the initial mass flow rate and velocity to zero with no gravity. The
    boundary conditions at the inlet and outlet are also set to zero, simulating
    a closed system. A region of high pressure is defined for one half of the
    domain, and a region of low pressure for the second half. An imaginary
    diaphragm divides the two regions before the simulation, and at $t=0$
    disappears.
    
    \begin{figure}[!h]
    	\centering
    	\includegraphics[width=0.55\textwidth]{images/Verification_Problem2_shock_tube}
    	\label{fig:Verification_2}
    	\caption{Setup for the shock tube problem}
    \end{figure}
    
    The problem was set up using the parameters given in table
    \ref{table:ST_Params}. The geometry is similar to the previous problem, but
    the length of the tube was elongated to take into account the faster
    propogation of the rarefaction and compression waves. The temperature and
    pressure of the air are near standard conditions.
    
    \begin{table}[!h]
    	\center
    	\label{table:ST_Params}
    	\caption{Input Parameters for Shock Tube Verification Probelm}
    	\begin{tabular}{|c|c|c|}
    		\hline
	    	Length 	  				&  25.00	& m 		\\ \hline 
	    	Channel Area			&  0.0001	& $m^{2}$	\\ \hline
	    	Wetted Perimeter		&  0.040	& m			\\ \hline
	    	Initial Pressure  		&  1.00		& bar		\\ \hline
	    	Initial Enthalpy		&  304.66575& kJ/kg		\\ \hline
	    	Initial Mass Flow Rate 	&  0.000	& kg/s		\\ \hline
	    	Initial Pressure Drop	&  0.09576  & bar 		\\ \hline
    	\end{tabular}
    \end{table}    
    
    Each region has a unique density corresponding to the different pressures
    using the equation of state for air given in equation \ref{eq:EOS_air} where $\gamma
    = 1.4$ is the ratio of specific heats for air. Additionally, the specific
    heat $Cp = 1.005 \frac{kJ}{kg-K}$ to convert between enthalpy and
    temperature. The initial enthalpy of the system is constant, but will change
    non-uniformly as a function of time. The velocity is initially set to zero,
    but will change as the compression and rarefaction waves move. Since the
    velocity is set to zero initially, it can't be used to evaluate the time
    step size. Instead the speed of sound can be evaluated using equation
    \ref{eq:sound_speed}, and this velocity can be used to calculate the time
    step. While there might be some slight change in the speed of sound due to
    enthalpy changes, it should remain effectively constant.
    
    \begin{equation}
    \label{eq:EOS_air}
    	\rho = \left( \frac{\gamma}{\gamma - 1} \right) \frac{P_{abs}}{h} 
    \end{equation}
    
    \begin{equation}
    \label{eq:sound_speed}
    a = \sqrt{ \left( \gamma - 1 \right) h }
    \end{equation}
    
    \begin{equation}
    \gamma = \frac{C_{p}}{C_{v}}
    \end{equation}
    
    \subsubsection{Analytical Solution}
    
    When the diaphragm disappears, a compression wave will move to the right and a
    rarefaction wave to the left. These two waves split the domain into four
    distinct regions. There is a region to the left of the rarefaction wave that has
    the same properties as the initial left region. There is a region between the
    rarefaction wave and the initial location of the diaphragm. There is a region
    between the initial location of the diaphragm and the compression wave. There is
    a region to the right of the compression wave, that has the same properties
    as the initial right region. The analytical solution does not take into
    account reflection off of the walls, however the numerical solution can due
    to the applied boundary conditions for mass flow rate.
    
    \begin{figure}[!h]
    	\centering
    	\includegraphics[width=0.65\textwidth]{images/Shock_Tube/Shock_Tube_regions}
    	\label{fig:V2_pressure_scaling}
    	\caption{Regions within the shock tube based on rarefaction and compression}
    \end{figure}
    
    For a perfectly caloric gas, the following equations are provided
    \cite[p. 238]{McGraw-Hill} given the initial conditions for state 1 and
    4 in conjuction with equation \ref{eq:sound_speed}. An iterative method is required
    to solve \ref{eq:ref1_7.94} for $\frac{P_{2}}{P_{1}}$. Once the region
    properties are obtained, the regions themselves are mapped by comparing the
    current position and time to the velocity of the rarefaction and compression
    wave. 
    
    \begin{equation}
    \label{eq:ref1_7.94}
    	\frac{P_{4}}{P_{1}} = \frac{P_{2}}{P_{1}} \frac{
    	\left( 1 - (\gamma-1)(\frac{a_{1}}{a_{4}})(\frac{P_{2}}{P_{1}} -1) \right) }
    	{\sqrt{2 \gamma \left[ 2 \gamma + (\gamma +1)(\frac{P_{2}}{P_{1}} -1)\right]}}^{
    	- \frac{2 \gamma}{\gamma - 1}}
    \end{equation}
    
    \begin{equation}
    	\frac{T_{2}}{T_{1}} = \frac{P_{2}}{P_{1}} \left(
    	\frac{\frac{\gamma + 1}{\gamma - 1} + \frac{P_{2}}{P_{1}}}
    	{1 + (\frac{\gamma +1}{\gamma - 1}) \frac{P_{2}}{P_{1}}} \right)
    \end{equation}
    
    \begin{equation}
    	\frac{\rho_{2}}{\rho_{1}} =
    	\frac{1 + (\frac{\gamma + 1}{\gamma - 1}) \frac{P_{2}}{P_{1}}}
    	{\frac{\gamma + 1}{\gamma - 1} + \frac{P_{2}}{P_{1}}} 
    \end{equation}
    
    \begin{equation}
    	W = a_{1} \sqrt{ \frac{\gamma + 1}{2 \gamma} 
    	                 \left( \frac{P_{2}}{P_{1}} -1 \right) +1 }
    \end{equation}
    
    \begin{equation}
    	P_{2} = P_{3}
    \end{equation}
    
    \begin{equation}
    	\frac{P_{3}}{P_{4}} = 
    	\left( \frac{\rho_{3}}{\rho_{4}} \right)^{\gamma} =
    	\left( \frac{T_{3}}{T_{4}} \right)^{\frac{\gamma}{\gamma -1}}
    \end{equation}
    
    \subsubsection{Results and Error}
    
    Enthlapy and density have a discontinuity as seen in figure
    \ref{fig:V2_result_top} where the diaphragm was initially placed . Similarly
    there are discontinuities that move with the rarefacation and compression
    waves for pressure, density, and velocity. The largest error occurs around
    these discontinuities as seen in figure \ref{fig:V2_result_bottom}. A 
    fine spatial mesh and a small time step are needed to accurately reflect the
    exact solution.
    
    \pagebreak
    \begin{figure}[!h]
    	\centering
    	\includegraphics[width=1.30\textwidth,angle=90.0]{images/wave1_200dP_N1000/tmp/plot_shocktube_0034}
    	\caption{Comparison of analytical and numerical results for shock tube}
    	\label{fig:V2_result_top}
    \end{figure}
    
    \pagebreak
    \begin{figure}[!h]
    	\centering
    	\includegraphics[width=1.30\textwidth,angle=90.0]{images/wave1_200dP_N1000/tmp/plot_st_err_0679}
    	\caption{Truncation error for shock tube}
    	\label{fig:V2_result_bottom}
    \end{figure}
    
    \pagebreak
    \subsubsection{Scaling of Error}
    
    A Richardson Extrapolation was performed on the density, enthalpy, and mass
    flow rate for the shock tube analysis in time. The error approaches zero
    as the time step size gets smaller as seen in figure
    \ref{fig:ST_Err_rho}. The order of accuracy with respect to time is
    converges to first order accurate as the time step size decreases. The range
    of time step values shown are within the asymptotic limit. 
    
    \begin{figure}[!h]
    	\centering
    	\includegraphics[width=0.5\textwidth]{images/Shock_Tube/Difference_rho}
    	\caption{Richardson Extrapolation of the shock tube results N=50}
    	\label{fig:ST_Err_rho}
    \end{figure}
    
    \begin{figure}[!h]
    	\centering
    	\includegraphics[width=0.50\textwidth]{images/Shock_Tube/Temporal_Order_Of_Accuracy_rho}
    	\caption{Temporal Order of Accuracy for shock tube}
    	\label{fig:ST_OOA_rho}
    \end{figure}
	
	\clearpage
	\subsection{Conduction Equations} \label{Cond_Eqs}
	For this project, the equation for a 1-D slab \eqref{eq:heat_cond} was used
	instead of using cylindrical coordinates. 
	
	\begin{equation}
		M c_{p} \left( \frac{\partial T_{i,solid}}{\partial t} \right)
		- A \left( \frac{\partial k_{solid} T_{i,solid}}{\partial x}\right) 
		- Q 
		+ A_{surf} h_{ht} \left( T_{w,solid} - T_{liq}\right) 
		= 0
		\label{eq:heat_cond}
	\end{equation}
	
	The first term is the thermal mass of the control volume multiplied by the rate
	of change of the temperature.The next term describes the conduction to adjacent
	control volumes. If the cell face is at the center, then the gradient through
	the face is zero to account for symmetry within the control volume. The area is
	taken as the cross sectional area at the middle of the cell. The next term is
	the heat generated within the region. The last term is the heat transfer out of
	the region due to convection. If the cell is not adjacent to the fluid, then
	this term gets dropped by setting the heat transfer coefficient to zero. This
	same term must be added to the single phase liquid energy residual to conserve energy.
	
	\begin{equation}
		M_{i} c_{p_{i}}\left(T_{i}^{n}\right) 
		\left( \frac{\partial T_{i,solid}}{\partial t} \right) 
		- A \left( \frac{\partial k_{solid} T_{i,solid}}{\partial x}\right) - Q 
		+ A_{surf} h_{ht} \left( T_{w,solid} - T_{liq}\right) 
		= 0
		\label{eq:FD:heat_cond}
	\end{equation}
	
	\begin{figure}[!h]
		\centering
    	\includegraphics[width=0.65\textwidth]{images/Heat_Conduction_Jacobian_Diagram}
    	\caption{Jacobian matrix for fluid and solid residuals}
    	\label{fig:solid_liquid_jacobian}
	\end{figure}
	
	When the conduction equation \eqref{eq:heat_cond} is written as residual
	function, it can be appended to the Jacobian structure as shown in figure
	\ref{fig:solid_liquid_jacobian}. The original fluid jacobian matrix is in the
	top left and has a block entry for each fluid cell. The solid jacobian matrix
	is in the bottom right, where each row corresponds to a single solid energy
	balance. Each block in the diagram represents an axial level with 4 radial
	positions. The first entry only has two temperatures due to symmetry about the
	center. The last entry only has two entries since it is interacting with the
	fluid and the surface temperature is lagging. While this represents the full
	jacobian matrix, the heat conduction solution is not fully coupled with the
	single phase liquid solution. This can be seen by the lack of terms in the
	upper right and lower left quadrants.
