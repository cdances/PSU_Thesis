%SourceDoc ../YourName-Dissertation.tex
\vspace*{-80mm}
\chapter{Residual Formulation} \label{chapter3:residual_formulation}
	    
    A residual is simply the difference between the value at some future time
    $n+1$ and the value at the current iteration $k$. This can be applied to
    desired variables as shown in equations
    \eqref{eq:residual_def:P}, \eqref{eq:residual_def:h},
    \eqref{eq:residual_def:u}, and \eqref{eq:residual_def:rho}. Residuals can
    also be applied to the conservation equations by substituting the definition
    of the residual variables into the conservation equations. This will
    effectively change any variables evaluated at $n+1$ to $k$. Each cell will
    have three residual variables and three residual equations. For the entire
    solution, we will then have a residual variable array $\delta X$, and a
    residual function array $F(X)$ which defines a linear system as seen in 
    equation \eqref{eq:linear_system}.
        
    \begin{equation}
    	\label{eq:residual_def:P}
    	\delta P_{i} = P^{n+1}_{i} - P^{k}_{i}
    \end{equation}
    
    \begin{equation}
    	\label{eq:residual_def:h}
    	\delta h_{i} = h^{n+1}_{i} - h^{k}_{i}
    \end{equation}
    
    \begin{equation}
    	\label{eq:residual_def:u}
    	\delta u_{i+\frac{1}{2}} = u^{n+1}_{i+\frac{1}{2}} - u^{k}_{i+\frac{1}{2}}
    \end{equation}
    
    \begin{equation}
    	\label{eq:residual_def:rho}
    	\delta \rho_{i} = \rho^{n+1}_{i} - \rho^{k}_{i}
    \end{equation}
    
    \begin{equation}
    	\label{eq:linear_system}
    	J \delta X = - F(X)
    \end{equation}
    
    The Jacobian matrix is defined in equation \eqref{eq:jac_def} as the derivative
    of each response of the function $F_{j}$ with respect to each variable $X_{i}$.
    The derivative can be calculated numerically as shown by equation
    \eqref{eq:jac_numerical} where $\epsilon$ is a small numerical value. For
    COBRA-TF the equations are linear, and this numerical approximation
    of the Jacobian matrix is exact. This should produce the same jacobian
    matrix that COBRA-TF currently generates analytically. 
    
    \begin{equation}
    	\label{eq:jac_def}
    	J_{i,j}=\frac{ \partial F_{j}(X)}{\partial X_{i}}
    \end{equation}
    
    \begin{equation}
    	\label{eq:jac_numerical}
    	J_{i,j}  \approx \frac{F_{j}(X_{i}+\epsilon)-F_{j}(X)}{\epsilon}
    \end{equation}
    
    To build the jacobian matrix, an object oriented class was created that
    contains three arrays. An array that points to the residual functions, an
    array that points to the position within a target variable arrray, and an
    array that has the index that the function is to be evaluated at. These
    lists can be appended to in any order, but have to be appended all at the
    same time so that variables and functions must correspond with each other.
    Then to construct the jacobian matrix, the residual function and residual
    variable arrays can each be looped over to numerically build the jacobian
    matrix as seen in figure \ref{fig:Jacobian_Setup}. 
    
    \begin{figure}[!h]
    	\centering
    	\includegraphics[width=0.45\textwidth]{images/Jacobian_Setup}
    	\caption{Strucutre of the jacobian matrix for single phase liquid}
    	\label{fig:Jacobian_Setup}
    \end{figure}







